\documentclass[a4paper, 10pt, conference]{ieeeconf} % Use this line for a4 paper
\IEEEoverridecommandlockouts % This command is only needed if you want to use the \thanks command
\overrideIEEEmargins

% See the \addtolength command later in the file to balance the column lengths
% on the last page of the document

%\usepackage{fontspec}
%\setmainfont{MinionPro-Regular.otf}

% Support for changing between draft-style and published version.
\newif\ifdraft
%\drafttrue
\draftfalse

% Enable Æ
\usepackage[utf8]{inputenc}

% Enable strikethrough
\usepackage[normalem]{ulem}
% enable images
\usepackage{graphicx}


% Enable links
\usepackage{hyperref}

% Enable To-do notes
\ifdraft
	\usepackage[textsize=small]{todonotes}
	\setlength{\marginparwidth}{1.4cm} 		
\else
	\usepackage[disable]{todonotes}
\fi

% Support for drafting and sketching
\usepackage{xcolor}
\definecolor{dark-cornflower-blue-2}{RGB}{17,85,204}
\definecolor{dark-green-2}{RGB}{56,118,29}

\ifdraft
	\newcommand{\sk}[1]{{\color{dark-green-2}  #1}}
	\newcommand{\dr}[1]{{\color{dark-cornflower-blue-2} #1}}
	\newenvironment{sketch}{\color{dark-green-2}}{\ignorespacesafterend}
	\newenvironment{draft}{\color{dark-cornflower-blue-2}}{\ignorespacesafterend}
\else
	\newcommand{\sk}[1]{#1}
	\newcommand{\dr}[1]{#1}
	\newenvironment{sketch}{}{}
	\newenvironment{draft}{}{}
\fi

% Support for marking missing citations
\newcommand{\source}[0]{{\color{blue}\textsuperscript{[\textsf{\textit{need cit.}}]}}}


% Enable \cref
\usepackage{cleveref}

\usepackage[style=ieee,
            urldate=iso8601,
            maxnames=3,
            backend=biber]{biblatex}
            
% Settings for source code listings
\usepackage{listings}
\lstset{ %
  aboveskip=10pt,
  belowskip=4pt,
%  abovecaptionskip=0pt,
%  backgroundcolor=\color{white},   % choose the background color; you must add \usepackage{color} or \usepackage{xcolor}
  basicstyle=\fontsize{9pt}{10pt}\ttfamily,        % the size of the fonts that are used for the code
%  belowcaptionskip=0pt,
%  breakatwhitespace=true,         % sets if automatic breaks should only happen at whitespace
%  breakindent=17pt,
%  postbreak=\space\space,
  breaklines=true,                 % sets automatic line breaking
  captionpos=b,                    % sets the caption-position to bottom
%  commentstyle=\color{commentgray},    % comment style
%  deletekeywords={...},            % if you want to delete keywords from the given language
%  escapeinside={\%*}{*)},          % if you want to add LaTeX within your code
%  escapeinside={@}{@},
  frame=single,	                   % adds a frame around the code
%  framesep=3pt,
%  keepspaces=true,                 % keeps spaces in text, useful for keeping indentation of code (possibly needs columns=flexible)
%  keywordstyle=\color{black}\bfseries,       % keyword style
%  language=Haskell,                 % the language of the code
%  otherkeywords={Type,Vect,Nat,pure,pureM,Eff,Effect,MkEff,sig,Address,EFFECT,!,<$>,<*>,:,::},           % if you want to add more keywords to the set
  numbers=left,                    % where to put the line-numbers; possible values are (none, left, right)
  numbersep=7pt,                   % how far the line-numbers are from the code
  numberstyle=\tiny\ttfamily, % the style that is used for the line-numbers
%  rulecolor=\color{black},         % if not set, the frame-color may be changed on line-breaks within not-black text (e.g. comments (green here))
  showspaces=false,                % show spaces everywhere adding particular underscores; it overrides 'showstringspaces'
%  showstringspaces=false,          % underline spaces within strings only
%  showtabs=false,                  % show tabs within strings adding particular underscores
  stepnumber=1,                    % the step between two line-numbers. If it's 1, each line will be numbered
%  stringstyle=\color{stringmauve},     % string literal style
%  tabsize=2,	                   % sets default tabsize to 2 spaces
%  title=\lstname,                   % show the filename of files included with \lstinputlisting; also try caption instead of title
  xleftmargin=9pt,
%  xrightmargin=10pt
}

\addbibresource{references.bib}

\title{\huge DAB\\[0.5em] \large Decentralized Autonomous Bank\\[1em]\today \\[1em] v0.1 }

\author{Tao Feng\\ \href{}{}  
\and \\ \href{}{}
\and \\ \href{}{} 
}

 
\begin{document}
\maketitle

\begin{abstract}

\end{abstract}

%\pagebreak


\tableofcontents
\thispagestyle{plain} % adds numbering to pages
\pagestyle{plain} % adds with ^ numbering to pages

\section{Introduction}
A bank is a financial institution that pools social wealth and resources to make events. In a way, the banking system helps promote economic prosperity and assures assets safety: loaning starting capitals for start-ups and entrepreneurs, and at the same time generating interest for depositors. However, traditional structures and modes of economy have been changing with the advent of new technologies, like BlockChain and Smart Contracts. In recent years, people have gradually got accustomed to various types of virtual currencies and applications based on them, which have not got a sound and reliable platform like a bank to invest and earn profits yet. Thus, a call for banking systems of virtual currencies arises.
On the hand, traditional banks hold a large share of the profits, which should have belonged to both depositors and loanees. Besides, People are not contented with this hierarchical administration because of its low efficiency and manifold restrictions. The procedures of loaning take numerous risks assessment and audit work. These complicated and repetitive operations increase unnecessary costs both in labor and in material, adding to inconvenience of a loan. 
Therefore, we propose a self-governed banking system transplanted on BlockChain, naming \textbf{Decentralized Autonomous Bank}, \textbf{DAB} for short. 
This will be \underline{the first} crowdfunded Ethereum banking system on BlockChain in history. The main contributions of this program are as follows:

\begin{itemize}
   \item The proposed banking system is \textbf{crowdfunded} by common users rather than authorities. With BlockChain technology, data of transactions generated by users can be recorded more accurately, and meanwhile these records can neither be modified nor be checked by anyone, assuring its reliability and security.
   \item The proposed banking system transforms the abstract concept, "credit", into measurable units for new asset class of "\textbf{tokens}" that are typically issued in Initial Coin Offerings (ICOs for short) through Smart Contracts to cut out unnecessary procedures for assessment and approval.
   \item As the first \textbf{Ethereum bank} on BlockChain, users of which can enjoy relatively high interest when depositing their Ethereum in the bank and cheaper yet more convenient loaning services than one can do in actual banks.
\end{itemize}

The rest of this paper is structured as follows. Section 2 gives a detailed instruction of DAB. How the banking system will be crowdfunded, established and  finally put into operation is presented in Section 3. Based on the system, section 4 describes DAB's expected outcome in the market of Ethereum. Section 5 refers to recent work and our progress on DAB. Section 6 provides a list of terminologies concerned in the paper.

\section{Concepts and Functions}
As mentioned above, not only users can deposit, withdraw, lend, loan or repay Ethereum more cost-effective in this crowdfunded DAB, but also procedures on risks assessment and credit approval are simplified. To realize these regular functions, a group of new concepts are needed in this banking system, which contains four types of tokens, two sub-banks and two main contracts.

\subsection{Tokens}
\textbf{DepositToken} (\textbf{DPT} for short) is a type of token for depositing function, while \textbf{CreditToken} (\textbf{CDT} for short), \textbf{SubCreditToken} (\textbf{SCT} for short) and \textbf{DiscreditToken} (\textbf{DCT} for short) are collectively referred to for loaning function under a joint name, \textbf{Generalized Credit Token}.

\begin{itemize} 
   \item \textbf{DepositToken (DPT)}: a type of token for Ethereum deposited into the bank, which represents a share one holds per token in the pool of reserve-deposit funds. DPT is negotiable in the market and can be either transferred or cashed.
   \item  \textbf{CreditToken (CDT)}: a type of token for Ethereum that one user can loan from the bank, which represents a share one holds per token in the pool of reserve-credit funds and one user's credit ceiling. CDT is negotiable in the market and can be cashed from CreditAgent Sub-Bank without any fees.
   \item  \textbf{SubCreditToken (SCT)}: the secondary form of CDT in the process of a loan, which represents a condition in debt. If the loan is repaid in time, SCT will be elevated back to CDT; if not, the SCT will be further degraded to DCT, whose value is far less than its original form. SCT is an non-negotiable token in the market, with less value than CDT, and can be neither transferred nor cashed.
   \item \textbf{DiscreditToken (DCT)}: the tertiary form of CDT if one's debt is overdue, which indicates an overdue debt.  It is of less value than SCT. DCT decays with time, but the remnants can be elevated back to CDT once the debt is paid off anytime. Although DCT can not be cashed, it can be transferred to another user with a certain number of fees, if the user is willing to repay the loanee's debt.
\end{itemize}

\subsection{Sub-Banks}
DAB is comprised of two independent sub-banks, which are DepositAgent Bank and CreditAgent Bank. For each bank, there is a independent pool of reserve funds in it.
\begin{itemize} 
   \item \textbf{DepositAgent Bank}: a sub-bank where users deposit their Ethereum getting DPT as token, and withdraw their Ethereum with their tokens. The sum of the Ethereum and DPTs deposited are called a pool of reserve-deposit funds. 
   \item \textbf{CreditAgent Bank}: a sub-bank where users loan and repay Ethereum, and gain bonus CRT as reward. The sum of the Ethereum and CRTs in this sub-bank is called a pool of reserve-credit funds.
\end{itemize}

\subsection{Contracts}
The two main contracts are DepositAgent Contract and CreditAgent Contract, which are responsible for the bank's depositing and loaning function, respectively.

\begin{itemize} 
   \item \textbf{DepositAgent Contract}: a contract in accordance with which deposit-related behavior occurs, such as users depositing their Ethereum into DepositAgent Bank, withdrawing Ethereum with DPT, transferring DPT to other users, etc. The contract also stipulates that the reserve-deposit ratio is automatically adjusted with the variation of negotiable DPTs in the market, thereby calculating the price of a DPT at a certain point.
   \item \textbf{CreditAgent Contract}: a contract in accordance with which credit-related behavior occurs, such as users loaning Ethereum from CreditAgent Bank with CDT, repaying Ethereum, cashing CDT, gaining bonus CRT, CRT degrading, DCT transferring and decaying, etc. The contract also stipulates that the price of a CDT is dependent on the amount of the reserve-credit funds, the reserve-credit ratio and the number of negotiable CDTs in the market.
\end{itemize}

\subsection{Behaviors}
This sub-section focuses on what users can do in DAB. The main four behaviors that users may carry out are listed below.

\begin{itemize}
   \item Deposit and withdraw. For one thing, a user can deposit Ethereum into DepositAgent Bank, get equivalent DPTs as token for a certain share of the reserve-deposit funds, and enjoy interest from it. For another, user can withdraw Ethereum and return his/her DPTs back to the bank.
\end{itemize}

\section*{ACKNOWLEDGMENTS}

%\bibliographystyle{plain}
%\bibliography{references.bib}

%\addcontentsline{toc}{chapter}{References}
%\input{include/backmatter/References}
%\bstctlcite{IEEEexample:BSTcontrol}
\printbibliography



\end{document}