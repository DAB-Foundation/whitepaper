\documentclass[a4paper, 10pt, conference]{ieeeconf} % Use this line for a4 paper
\IEEEoverridecommandlockouts % This command is only needed if you want to use the \thanks command
\overrideIEEEmargins

% See the \addtolength command later in the file to balance the column lengths
% on the last page of the document

%\usepackage{fontspec}
%\setmainfont{MinionPro-Regular.otf}

\input{settings}

\addbibresource{references.bib}

\title{\huge DAB\\[0.5em] \large Decentralized Autonomous Bank\\[1em]\today \\[1em] v0.1 }

\author{Tao Feng\\ \href{}{}  
\and \\ \href{}{}
\and \\ \href{}{} 
}

 
\begin{document}
\maketitle

\begin{abstract}

\end{abstract}

%\pagebreak


\tableofcontents
\thispagestyle{plain} % adds numbering to pages
\pagestyle{plain} % adds with ^ numbering to pages

\section{Introduction}
\subsection{A Call for Monetary Liberation}
A bank is a financial institution that pools social resources to make events. However, the existence of the bank must have its value and significance. In a way, banking system helps promote economic prosperity and assures safety of assets by pooling social wealth and resources: providing starting capitals for start-ups and entrepreneurs, at the same time generating interest for other deposit users by obtaining interest from the loan. But most business banks are regulated and controlled by central banks of authorized governments, thus rendering a hierarchical system. Moreover, traditional banks hold a large share of the profit except their employees' salaries, which should have belonged to both deposit users and loan users. Besides, People are not contented with this centralized administration because of its low efficiency and manifold restrictions. Thus, a call for balance between monetary liberation and banking functions maintenance arises.
\subsection{Beyond an Age of Old}
With the advent of new technologies, structure of the society and how it works have become to reform. The block chain technology is created to eliminate the highly centralized administrations like banks. Traditionally, banks provide two main types of service for customers:

\begin{itemize} 
   \item Deposit spare cash to obtain steady interest;\label{}
   \item Withdraw money from the bank account for living expenditure;\label{}
   \item Loan money for urgency and pay it back with extra interest.\label{}
\end{itemize}

However, the service procedures take numerous risks assessment and audit work. These complicated and repetitive operations increase unnecessary costs both in labor and in material, adding to difficulties of loaning without avoiding possibilities of bad debts. Therefore, an assumption is proposed that banking systems are transplanted on the block chain. The functions of the systems are retained, but all the profits that banks would have gained now belong to the users, thus increasing a higher deposit interest rate and meanwhile lowering users' loaning interest rates.
Based on the distributed general ledger technology, this assumption needs a certain type of digital money as a substitute for real money to circulate and function. Although bitcoin is the first successfully encrypted currency, there is only one account book that records each transaction. As the currency with the largest market and community, Ethereum supports smart contract, enabling individuals to publish their own tokens on the platforms. That is what makes Ethereum the best choice for the establishment of our decentralized autonomous bank. With the emergence of block chain technology, data of transactions generated by users can be recorded more accurately, and at the same time the records on the block chains cannot neither be modified nor be checked by anyone. 
To realize the decentralized autonomous bank, we transform the traditional abstract concept of "credit" into measurable units for new asset class of "tokens" that are typically issued in Initial Coin Offerings (ICOs for short) through smart contracts. Credit is an abstract concept yet a medium for banks and individuals to safely trade with each other. Yet it itself is not a tradable commodity, but an attribute of a certain individual, organization or institution. Once credit becomes measurable and negotiable Generalized Credit Token, a credit market will come into being and thus the value of credit will be guaranteed.
Therefore, a credit-authorization-simplified yet credit-security-guaranteed banking system, a decentralized autonomous bank, DAB for short, is proposed.

\section{What is Decentralized Autonomous Bank?}
On platform of DAB, not only can users gain profits through depositing, but they can enjoy relatively lower interest of loaning service provided accordingly.
These functions of DAB are mainly realized by a main contract and four types of tokens. The main contract contains two sub-contracts, which is responsible for depositing and loaning, respectively. As regards with the tokens, DPT (Deposit Token) is for depositing function, while CDT (Credit Token), SCT (Sub-credit Token) and DCT (Discredit Token) are collectively referred to for loaning as a joint name, Generalized Credit Token.
\subsection{Contracts}

\begin{itemize} 
   \item \textbf{DepositAgent Contract}: contract for deposit reserve fund for deposit user. It can automatically adjust the reserve rate of the deposit contract according to the amount of the deposit point outside the contract, and calculate the price of the corresponding deposit point.
   \item \textbf{CreditAgent Contract}: contract to deposit credit point reserve of credit users. It can guarantee the value of the credit point. The price of the credit spot calculates the price of the credit according to the reserve amount, the reserve ratio and the current amount of credit points (including the credit point, the secondary credit point and the break point). And be able to open up the credit spot.
\end{itemize}

\subsection{Tokens}

\begin{itemize} 
   \item \textbf{Deposit Token (DPT for short)}: a token in accordance with the ERC20 standard and a negotiable unit of depositing behavior.
   \item  \textbf{Credit Token (CDT for short)}: a token in accordance with the ERC20 standard and a negotiable unit to measure users' credit and their loaning allowance. It can be cashed from the contract without any fees.
   \item  \textbf{Sub-Credit Token (SCT for short)}: a token in accordance with the ERC20 standard and an non-negotiable secondary form of CPT when it is used for loaning. If the loaning is paid back in time, SCT will be transformed back to CDT by DepositAgent Contract; if not, it will be destroyed and users will lose the value of their credit.
   \item \textbf{Discredit Token (DCT)}: a token in accordance with the ERC20 standard and as alternative form of SCT when users do not have the ability or are not willing to pay back their loans. Users have to actively transform SCT into DCT through the contract, or the overdue SCT will be destroyed. It cannot be cashed, but can circulated with a certain number of fees.
\end{itemize}

\subsection{Operations}
Users have the following four operations on the platform of DAB.

\begin{itemize}
   \item Deposit money in accordance with DepositAgent Contract. According to the contract, users deposit Ethereum into the DepositAgent Contract Bank and enjoys interest from it. Moreover, the deposit can be cashed at any time.
   \item Loan money in accordance with CreditAgent Contract. According to the contract, users can establish an Ethereum loaning agreement with an appointed user. First, users who want to loan money have to exchange their CDTs to the sub-contract bank for equivalent value of Ethereum and prepay a certain amount of interest based on their loaning time. The prepaid interest will be used for issuing new CDTs at four times of the price of ICO and then the new CDTs will be rewarded to the users who have paid the interest previously. At the same time, by CreditAgent Contract, the exchanged CDTs will be transformed into equal numbers of SCTs back to the users. Finally, the users have to return the SCTs and the loan principal to the contract in a given period of time, which will trigger the  transformation of SCTs back to CDTs. Besides, users gain bonus CDTs as mentioned above. But if not returned timely,  the SCTs will be further degenerated into DCTs, the number of which will reduce as the time goes. The loss of DCTs means the loss of equal number of CDTs or SCTs.
   \item Be a guarantor of someone's loaning operation in accordance with CreditAgent Contract. According to the contract, a user can establish a new loaning agreement as a guarantor of another user who is in debt. The slight difference from the general loaning operation above is that the guarantor takes the risks.To be specific, it is the guarantor's CDTs that will be transformed into SCTs in this transaction, while the equivalent Ethereum (a net of deductions for the prepaid interest) will be given to the user in debt. Moreover, the CDT reward for retuning will be divided evenly for these two users. That is to say, the guarantor needs to secure the loaning agreement. If the user in debt cannot make an repayment in time, the guarantor shall undertake the repayment obligation.
   \item Found lending companies based on the third operation. If a user has adequate CDTs, he/she can offer loaning services with lower interest for other users. In this way, users in lending company can not only enjoy higher profit than that of DepositAgent Contract, but also gain bonus CDTs in this transaction.
\end{itemize}

\subsection {For users who loan}
Users can not loan money unless they have equivalent CDTs. To obtain CDTs, one has three ways: purchase some from the market, borrow some from a friend or entrust a third party with the loaning. 
Unlike traditional loaning services, DAB substitute CDTs for laboursome supervision and time-consuming examination, thus lowering cost for management and interest of a loan, which is mainly dynamically determined by the market and gross lending. Annual interest rate can be close to 4\% at a lower level of gross lending, and may reach 10\% at a higher level. However, it does not imply that the interest rate will grow without any limits. The lending will be inhibited at two thirds of the reserve level. 
With the help of a friend or a third party, a user without CDTs is also able to loan: commissioning the friend or the third party to loan money for the user. This relies on a creditable hypothesis that users with CDTs is likely to gain profits using their idle CDTs. Once their friends have needs of loaning, they will be willing to help: using their CDTs to loan and paying the corresponding interest. In this way, as long as the user obeys the loaning agreement, both he/she and his/her friend or the third party can also gain bonus CDTs out of it. Compliance with credit is beneficial to all. Through these operations, circles of loaning and lending have gradually come into being, users of which pass idle CDTs on to their trusted friends. Thus, the healthy mode will eventually lead to an increase in the number of CDT issuance and loaning reserve, and a gradual expansion of the market. The market will not get too large in the presence of users' withdrawals.

\section{Implementation}
The realization and implementation this decentralized autonomous bank have to illustrated separately according to different stages and operations.

\subsection{Modeling and Operations}
As shown in the figure, the blue curve represents the functional relation between \textbf{Cash Reserve Ratio (CRR)} and \textbf{Negotiable DPTs}. $a$, $b$, $l$, $d$ in the figure serve as parameters to adjust the shape and the position of the CRR curve (formula (\ref{CRR1})); the purple curve represents the functional relation between \textbf{Issue Price of DPT (IP)} and \textbf{CDT Issuance Number} (formula (\ref{IssuePrice3}); the green curve represents the functional relation between \textbf{Withdraw Price of DPT (WP)} and \textbf{DPT Issuance Number Before Activation, $x$} (withdrawal of DPTs is limited before activation, so ${ DPT Issuance Number = Negotiable DPTs}$) (formula (\ref{WithdrawPrice4})). CRR(DPT, x) represents the reserve ratio of DepositAgent Contract, meaning that \emph{users' deposits (Ethereum)} are deposited by CRR (DPT, x) into the DepositAgent Contract Bank and serve as the depositing reserve. Meanwhile, CDTs with equivalent value will be issued. The remaining ${1 - CRR(DPT, x)}$ \emph{users' deposits} are used as the loaning reserve, issuing equivalent CDTs. 

\begin{equation}\label{CRR5}
{CRR(CDT) = 3}
\end{equation}

\begin{equation}\label{Issue7}
{Issue(CDT) = \frac{1 - CRR(DPT, x)}{2 * Price(Initial)}}
\end{equation}

\subsection{Mintage and Credit}
\subsubsection{Before DepositAgent Contract Activation}
The two contracts are inactivated in the beginning. In this stage, users deposit Ethereum into the DepositAgent Contract Bank to obtain DPTs and CDTs (their initial prices are 1000 DPT/ETH and 1000 CDT/ETH, respectively), and to control the deposit withdrawals, the monetary issue of DPTs and CDTs are the exactly the same. The issue price of a DPT is calculated dynamically through formula (\ref{ IssuePrice3 }), while that of a CDT remains unchanged. The issuance of CDT is based on the formula (\ref{Issue7}). According to formula (\ref{CRR5}), the reserve-loaning ratio is 3. As issuance of DPTs increases, the price of one will rise, thus tokens gained by the same amount of deposit will be accordingly reduced. With the increase in the overall issuance number of DPTs, CRR (DPT) will decrease, thereby increasing the issuance of CDTs. The issuance number of both DPTs and CDTs are calculated through the overall issuance number. In a lower issuance number of DPTs, reserve-deposit rate CRR (DPT) is high (close to \emph{a}), the issuance number of DPTs \emph{x} is large, and that of CDTs is small. When the issuance number of DPTs increases, the deposit reserve increases accordingly, and then the reserve-deposit rate CRR (DPT) reduces (until to \emph{b}), the issuance number of DPTs decreases, and that of CDTs increases. The blue curve in the figure represents the change in CRR(DPT) as the overall issuance number. 
The green curve represents the change in the price of a DPT as $x$ changes. When ${x = x_0}$ DPTs are issued, the area bounded by the blue curve, the  $x$ axis, the  $y$ axis, and the line ${x = x_0}$ represents the reserve in DepositAgent Contract, and the current DPTs issuance ratio represents the instantaneous reserve rate at ${x = x_0}$. The area of the blue curve, the ${y = 1}$ axis and the line of ${x = x_0}$ represents the reserve fund in the CreditAgent Contract Bank. The reserve-loaning rate is 3 (more than 1), so when issuing CDTs, each ETH into the CreditAgent Contract Bank corresponds to ${\frac{1 - CRR(DPT, x)}{3\*Price(Initial)}}$ CDTs. But in order to ensure the issuance of CDTs, formula (\ref{Issue7}) is set as the implemented regulation of CDTs issuance. Before activation, users purchase each DPT at a lower price earlier yet they gain less CDTs as well, while users who purchase later have to cost more on each DPT yet they gain more CDTs than the early birds . Hence, depositors benefit no matter they are an early participant or not.
\subsubsection{After DepositAgent Contract Activation}
There are only two stages, which are before the activation and after of activation. Commonly, the contract has no termination. The contract can continue to issue new CDTs and DPTs when the deposit are allowed to be withdrawn. The profit of issuing new DPTs and CDTs is larger as long as no DPTs is left in the deposit sub-contract. But this does not imply that users can issue new tokens as they wish, for  the withdrawal behavior will not exchange all he DPTs in the contract. This rule can prevent the issuance of unnecessary new DPTs and CDTs. The issuance number of DPTs is entirely determined by the market, which has no upper limit. Once a DPT is issued, it can not be destroyed.
The green curve in the figure is the relationship between the price of DPT and the issuance number of it before activation. In order to suppress excessive market deposits, the price of DPTs increases slower and even decreases as the issuing price increases. The amount of each single deposit and withdrawal has its maximum limit, and higher fees for higher amount of deposits or withdrawals are set to avoid malicious behaviors by some users. The price of withdrawing CDTs is commonly a fixed value, in which ${CRR(CDT) = 3}$.

\begin{equation}\label{Balance8}
{Balance(CDT) = 2 * Supply(CDT)}
\end{equation}

In formula (\ref{Balance8}), the reserve for CDTs is two times the amount of its issuance. The issuance of a CDT is mainly accompanied by the process of mintage. But a CDT can be destroyed when users who hold CDTs by simple cash withdrawals or violating the contract. New CDTs can also be issued when the loaning operations expand. A complete loaning and successful transaction will reward CDTs equivalent to the value of a quarter of interest prepaid at the initial issuance price back to the user who obey the contract, after returning the principal and interest.

\subsection{Depositing and Withdrawal}
\subsubsection{Before DepositAgent Contract Activation}
Before the contract is activated, any cash withdrawal operation can not be satisfied. But the transfer operation of DPTs are not restricted. When the issuance number of DPTs exceeds the Issuance activation limit \emph{l} or the projected activation time is up, users can withdraw their DPTs to cash.

\begin{equation}\label{CRR1}
{CRR(DPT, x) = a * \frac{1}{1+e^(\frac{x - l}{d})} + b (0 < b < a < 1)}
\end{equation}

\begin{equation}\label{Price2}
{Price(Initial) = 1/100}
\end{equation}

\begin{equation}\label{IssuePrice3}
{IssuePrice(DPT) = \frac{1}{CRR(DPT, x)} * Price(Initial)}
\end{equation}

\begin{equation}\label{Issue6}
{Issue(DPT) = IssuePrice(DPT)}
\end{equation}

Before the activation, anyone can deposit within the upper limit 300 ETH, which ensures the stability of DPT's price and the CRR(DPT) not to change dramatically. In this way, the situation where users benefit from large withdrawals afterwards are avoid. The issuance price of DPTs is calculated in accordance with the formulas above. Small impact on CRR of each transaction ignored, the larger a single deposit is (within the limit), the greater the user can benefit.

\subsubsection{After DepositAgent Contract Activation}

After the activation, user can withdraw the tokens into Ethereum immediately. Each withdrawal operation will transfer the corresponding DPTs to an address in the contract. As long as there are some DPTs left in the contract, other users who deposit will first use those DPTs instead of issuing new ones. 
The projected activation time is two weeks long. Once the deposit reserve is higher than ${l + 2 * d}$ within two weeks, the sub-contract will be immediately activated. If the reserve does not reach ${l + 2 * d}$ but $l$ within two weeks, the contract will be activated as projected. If the reserve does not reach $l$ but ${l - d}$ within two weeks, the activation time will be extended another two weeks. If the reserve is less than ${l - d}$, that means the contract activation fails and a refund interface will be opened for users to redeem their initial capital (through a non-negotiable record called \textbf{Issuer Token}, either for refund after the activation failure or as commemorative coins after success).
Once the activation succeeds, the contract runs on line. The price of DPTs withdrawal is calculated by formula (\ref{WithdrawPrice4}). $x$ represents the negotiable numbers of DPTs, which equals the subtraction of the issuance number of DPTs and the number of DPTs in the contract. The price fro withdrawal is always lower than its issuance price so as to protect the interest of all the depositors from excessive dilution of the depositing reserves.

\begin{equation}\label{WithdrawPrice4}
{WithdrawPrice(DPT) = \frac{Balance(DPT)}{Supply(DPT) * CRR(DPT, x) * Price(Initial)}}
\end{equation}

\subsubsection{Mechanisms}
After the activation, mechanisms for calculating the price of depositing and withdrawing DPTs are as followed:
\subsubsubsection{Under Sufficient DPTs}
\subsubsubsubsection{Principles and Rules}
The price of a DPT is expected at a rather higher level in calculation so that users get less DPTs than they do in reality. The market price of a DPT after depositing behavior should be slightly lower than that when depositing in calculation. The price rises after depositing behavior, so ${Price(Low) = Price(Before)}$;${Price(Before) < Price(After)}$; reserve increases; reserve ratio drops. The difference between the calculation value and the actual value of DPTs users gaining acts as fees for all the DPT holders.
\subsubsubsubsection{Procedures}

\begin{itemize}
   \item Assuming that a user deposits $d$ ETH: ${Deposit(Max) = \frac{d}{Price(Low)}}$, calculate the maximun DPTs are needed for the depositing users.
   \item Calculate the amount of the reserve after depositing: ${b + d}$.
   \item Calculate the minimum reserve ratio: ${f(s + Deposit(Max))}$, among which $s$ represents negotiable DPTs outside the contract.
   \item Calculate the maximum price of a DPT: ${Price(High) = \frac{b + d}{s * f(s + Deposit(Max))}}$.
   \item Calculate the actual price: ${Price(Actual) = Price(High)}$.
   \item Calculate the actual number of DPT for depositing users: ${Deposit(Actual) = \frac{d}{/frac{b + d}{s * f(s + Deposit(Max))}}}}$.
\end{itemize}

\subsubsubsection{Under Insufficient DPTs}
\subsubsubsubsection{Principles and Rules}
Not only depositing behaviors, but also status of mintage and credit will be affected when Deposit Agent Contract Bank runs out of DPTs.
When exchanging the remaining DPTs from the bank, the price of a DPT is expected at a rather higher level in calculation so that the theoretically obtained DPTs by depositing users are less than they do in reality, which is consistent with the situation when DPTs in the bank is sufficient. Therefore, the price of a DPT after depositing should be slightly lower than that during calculation. 

After depositing, the price rises, thus making ${Price(Low) = Price(Before)}$, ${Price(Before) < Price(After)}$, reserve increase and reserve ratio drop.
On the other hand, a portion of users' deposit is used for mintage and credit. The difference between the calculation value and the actual value of DPTs users gaining acts as fees and mintage costs for all the DPT holders and developers. Consequently, the first step of the solution is to work out how large the portion of deposited Ethereum is that the remaining DPTs can cover. Then deduct the difference from the actual value to obtain the answer. Mintage increases reserve, lowers reserve ratio, and multiplies the number of negotiable DPTs. The reserve ratio will not drastically decrease when negotiable portion of DPTs is large. But the mintage behavior suppress the price of a DPT, for the growth ratio of negotiable portion remains unchanged. The turning point lies somewhere around ${l + 4d}$. This mechanism can rapidly dilute the value of DPT, and meanwhile quickly increase the total amount of reserve-loaning. To some extent, the over-mintage and over-crediting are inhibited as well. Because of a pretty high costs for mintage and crediting, the issue price of a DPT is much higher than the that in the bank when withdrawn (the costs of mintage is 30\% for each DPT or CDT).

\subsubsubsubsection{Procedures}

\begin{itemize}
   \item Assuming that DPTs are sufficient, then by the method above, ${Deposit(Actual)}$ is calculated to be greater than the remaining DPTs in the bank ${si}$.
   \item The maximum value of increased negotiable DPT numbers can then be calculated as ${Deposit(Max) = Deposit(Actual)}$. For the sake of higher costs for mintage in reality, the actual value is smaller than ${Deposit(Max)}$.
   \item Calculate the maximum value of reserve-deposit: ${b + d}$.
   \item Calculate the minimum reserve-deposit ratio: ${f(s + Deposit(Max))}$.
   \item Calculate the maximum price of a DPT: ${Price(High) = \frac{b + d}{s * f(s + Deposit(Max))}}$.
   \item Calculate the maximum value of deposited Ethereum that can be covered: ${Price(High) * si}$.
   \item Calculate the surplus Ethereum that should be used for mintage: ${d - Price(High) * si}$.
   \item If the Ethereum used for mintage is less than zero, the gap then will just be missed by the contract, which guarantees users obtaining the remaining DPTs. If it is greater than zero, then new DPTs and CDTs will be issued to users according to rules of mintage, i.e. users obtains  ${si + Issue(DPT, d - Price(high) * si)}$ DPTs and ${Issue(CDT, d - Price(High) * si)}$ CDTs.
\end{itemize}

\subsubsubsection{Withdrawing DPTs}
\subsubsubsubsection{Principles and Rules}
The price of a DPT is expected at a rather lower level in calculation thus rendering the price after withdrawal slightly higher than that in calculation. After withdrawal behavior, the price of a DPT declines, making ${Price(High) = Price(Now)}$, ${Price(Before) > Price(After)}$, reserve decline, and reserve ratio increase. The difference between the calculation value and the actual value of DPTs users gaining acts as fees for all the DPT holders. 

\subsubsubsubsection{Procedures}

\begin{itemize}
   \item Calculate the amount of Ethereum that should be given to withdrawal users, assuming that a user withdraws $x$ DPT: ${Withdraw(Max) = x * {Price(High)}}$.
   \item Calculate the minimum reserve after withdrawal: ${b - Withdraw(Max)}$.
   \item Calculate the reserve ratio after withdrawal: ${f(s - x)}$, among which $s$ represents negotiable DPTs outside the contract.
   \item Calculate the minimum price of a DPT: ${Price(Low) = \frac{b - Withdraw(Max)}{s * f(s - x)}}$.
   \item Calculate the actual price: ${Price(Actual) = Price(Low)}$.
   \item Calculate the actual number of DPT for depositing users: ${Withdraw(Actual) = x * /frac{b - Withdraw(Max)}{s * f(s - x)}}}$.
\end{itemize}

\subsubsection{Simulation}
The simulation results show that the principles and rules above can be satisfied and run in practice.

\begin{equation}\label{Price9}
{Price(CDT) = \frac{Balance(CDT)}{Supply(CDT) * CRR(CDT)} * Price(Initial)}
\end{equation}

The price of withdrawing CDTs is calculated by formula (\ref{Price9}).

\subsection{Loaning and Repaying}
\subsubsection{Before CreditAgent Contract Activation}
The projected activation time of CreditAgent Contract is due at the end of the second week after the activation of DepositAgent Contract activated. Before the activation, neither CDTs can be withdrawn, nor loaning behaviors are allowed. But the transfer behavior of CDTs is not restricted. When the loaning subcontract is successfully activated, users can immediately withdraw their CDTs and conduct loaning operations.
\subsubsection{After CreditAgent Contract Activation}
After the activation, the number of tokens in the DepositAgent Contract Bank basically meet the demand of the market, so the loaning operations can then be allowed to conduct by users. On the one hand, CDTs can be withdrawn into Ethereum from the loan contract. On the other, loan of Ethereum can be obtained by pledging CDTs from the loan contract. By withdrawing, CDTs will be destroyed and the negotiable number of it decreases. By loaning, users enjoy relatively lower interest rates and meanwhile gain bonus CDTs. Certainly, if a user exceeds the repayment time, the collateral credit point will be destroyed at a certain rate. Even if the user returns the principal and interest afterwards, he/she can not get the same number of CDTs as he/she owned before.

\section{Risk Assessment}
As CDT holders, there are five strategies to participate in the development of DAB:

\begin{itemize}
   \item Cash CDTs into Ethereum in accordance with the contract;
   \item Refuse to abide by a loaning agreement;
   \item Abide by a loaning agreement;
   \item Be a guarantor or found a lending company, and lend Ethereum to others by CDTs;
   \item Sell CDTs to other users.
\end{itemize}

The former two strategies are regarded as breaking agreements, while the latter three are recommended according to the contract. The difficulty in implementation of lending and loaning on Block Chain is the chance that users may not abide by their agreement. CDT is necessary for loaning according to the contract, which lending Ethereum at the issuance price of the CDTs. Most CDTs are issued along with the mintage and issuance (also known as crediting) of DPTs. The more users deposit, the more CDTs will be issued in the market. At the same time, CDT works as the evaluation of a user's credit, assuring the value of the reserve-loaning. For this reason, a system for CDT accounting is then established , asking for pledging CDTs to conduct loaning behavior (transforming CDTs into SCTs).
By setting up CreditAgent Contract with its reserve ratio of 3 and a free market for CDTs, the price of a CDT is always higher than the available amount that each CDT can loan. That is to say, the mechanism raise the costs for breaking agreements to 'induce' users to pay their debts timely. With CRR over 1, the value of owning a CDT is higher than destroying one, so users incline not to break the agreement. As to breaking agreements, there are two ways to do that: one is withdraw CDTs from the CreditAgent Contract Bank; the other is not paying debts timely (turning CDTs into SCTs). The former way is realized in accordance with formula (\ref{Price9}), and then the withdrawn CDTs expire with a portion of their original value left to their owners. In the latter way, CDTs are first transformed into SCTs and then proceed to be transformed into DCTs with time. The DCTs will then be destroyed with time. Certainly, users can still re-transform DCTs or SCTs back into CDTs by paying their debts, even if it is overdue. Besides, per CDT destroyed or expired will raise the value of remaining ones due to ${CRR(DPT) > 1}$, which encourages users to own more CDTs rather than abandon them according to CreditAgent Contract. What the contract attempt to assure is that Ethereum gained by either withdrawing CDTs or untimely returning debts is set to be far less than that by owning or purchasing equivalent amount of CDTs.
There are only two ways of issuing new CDTs, along with the mintage of DPTs or by obeying loaning agreements, which is calculated by the interest paid by users. The interest paid by users are used for raising the value of both CDT and DPT. Thus, depositors are encouraged to hold DPTs, those who abide by loaning agreements are rewarded for keeping their words and those who break lost profits they could have earned.

\subsection{Stage of Market Agreement}
The market agreement stage refers to a few weeks after the activation and activation of the deposit contract. At the beginning of the loan contract, there may be too many credit points in the market, then the credit point will be destroyed by the normal withdrawal or default loan. Then the credit point will increase in price in the first way. The act of gradually suppressing intentional default loans.

\begin{itemize}
   \item Normal withdrawals. Less than default withdrawals, will also be smaller than its reserve value, but also smaller than the market price of the credit point of circulation, the direct cash is better than default cash, was suppressed in second ways.
   \item Default loan. Breach of the people gradually left the market, the total credit gradually reduced, a default would not deduct all the credit, but in repayment over time, gradually reduce the pledge of credit, will also end until the user interest. The default is equivalent to selling the value of the credit reserve to other people who hold the credit point, and the price of the direct cash will rise. The second way is also not encouraged. The value of a breach is lower than the value of the sale. So it's suppressed in fourth ways.
   \item Normal lending. Using a normal credit point of credit, 1 credit points are paid for each 1ETH interest. That is, the user converts the value of the interest ETH into the value of the credit point. This approach is encouraged. Half of the interest on the credit is deposited into the reserve pool of the loan, and half is returned to the depositor, the holder of the deposit point, as interest. If most people in the market are borrowing normally (complying with credit), an increase in the number of credit points will increase the size of the loans and encourage those who comply with the credit. If most people in the market default or make withdrawals, the value of the credit will gradually increase, thereby increasing the value of credit, thereby encouraging people to hold credit points.
   \item Establishment of sub banks. If a user or group of users has a large number of credit points, then they can spontaneously form a sub bank. The funds that are obtained from the central bank by using credit points, and then lend money to the users who believe it is reliable. This model will be a more widely adopted approach. The sub bank has a certain amount of the loan amount, or the right to examine and approve the third party loans. Users benefit by helping others get loans.
   \item Selling credit points. Because the credit through third ways to cash 1ETH can get the price in the market, so the credit market is at least 1ETH, and credit is worth more than the value of 2ETH, so you don't tend to use first, second ways to cash, and credit transfer to 0.05 of the fee, so the point is not to encourage credit transfer, the transfer fee is equivalent to comply with the credit people incentives, while improving the bad faith or unknown person credit admission cost. The price of a normal transaction on the market is higher than the first and second price (about its reserve value).
The number of credit points is determined by the size of the loan market, and if the demand for borrowing is high on the market, then the credit point will work well. Contract lending continues to earn interest and increase the value of credit points. If the market demand for borrowing is not high, then the credit point can not play a good role. Contract lending is small, making credit points appreciate at a slower rate. Users may be more likely to use credit points as a currency rather than a debit certificate. The functional value of credit points drops, affecting the overall value of credit points. Two weeks ago in the contract after activation, the number of credit points not only increase, will gradually fit market demand for credit, but it is still possible to overissue, so when the loan contract function opening, the number of credit points will be reduced, the market demand further.

%\section*{APPENDIX}


\section*{ACKNOWLEDGMENTS}

%\bibliographystyle{plain}
%\bibliography{references.bib}

%\addcontentsline{toc}{chapter}{References}
%\input{include/backmatter/References}
%\bstctlcite{IEEEexample:BSTcontrol}
\printbibliography



\end{document}